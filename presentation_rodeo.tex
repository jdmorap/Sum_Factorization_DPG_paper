% This text is proprietary.
% It's a part of presentation made by myself.
% It may not used commercial.
% The noncommercial use such as private and study is free
% Sep. 2005 
% Author: Sascha Frank 
% University Freiburg 
% www.informatik.uni-freiburg.de/~frank/
%
% additional usepackage{beamerthemeshadow} is used
\documentclass{beamer}
\usepackage{beamerthemeshadow}
\usepackage{hyperref}
\usepackage{color}
% \usepackage{datetime}
% \usepackage[useregional]{datetime2}
% \usepackage[USenglish]{babel}
% \newcommand{\mydate}{\formatdate{3}{4}{2017}}


\title[Fast quadrature for DPG]{Fast quadrature implementation for Discontinuous Petrov Galerkin (DPG)}
\subtitle{ }
\author{Jaime David Mora}
\institute{ICES - The University of Texas at Austin}
\date[FE Rodeo 2017] % (optional)
{Finite Element Rodeo, March 3-4 2017}
\titlegraphic{\includegraphics[width=0.5\textwidth,height=.1\textheight]{ices_fig.eps}}

\DeclareMathAlphabet{\mathpzc}{OT1}{pzc}{m}{it}
\begin{document}
% \title{Fast quadrature implementation for Discontinuous Petrov Galerkin (DPG)}
% \author{Jaime Mora}

\frame{\titlepage} 

\frame{\frametitle{Table of contents}\tableofcontents} 


\section{Motivation} 
\frame{ \frametitle{Sum factorization}
\begin{itemize}
\item In tensor-product 3D shape functions (e.g. hexahedral elements) decomposition into their 1D generating functions provides the chance to reorganize the quadrature and thus reduce the computational cost.
\item The theoretical cost improvement consists of going from $\mathpzc{O}((p+1)^9)$ to $\mathpzc{O}((p+1)^7)$ FLOPs.
\item This technique requires a symmetric setting in order to be taken profit the most. This is because a single call to each of the 1D-associated shape function subroutine is needed to get both the trial and test components of any matrix entry.
\item In DPG, the Gram matrix fulfills this condition, having the enriched test space functions both as test and trial components.
\end{itemize}
}


\section{Description of the sum factorization technique} 

\frame{\frametitle{Overview of sum factorization for Helmholtz equation}
 \footnotesize{The Helmholtz equation, in its discretized variational generates a matrix with the following entries:}
 \begin{figure}
  \centering
  \includegraphics[width=7cm]{matrix_entry_xi.png}  
 \end{figure}
 \footnotesize{where D is a symmetric matrix depending on the jacobian:}
\begin{figure}
  \centering
  \includegraphics[width=2.2cm]{defD.png}  
 \end{figure}
 \footnotesize{Integration of the matrix entries by quadrature:}
 \begin{figure}
  \centering
  \includegraphics[width=9cm]{quadrature.png}  
 \end{figure}
}

\frame{\frametitle{Overview of sum factorization for Helmholtz equation}
 \footnotesize{Classic algorithm of numerical integration:}
 \begin{figure}
  \centering
  \includegraphics[width=8.5cm]{classic_algo.png}  
 \end{figure}
}

\frame{\frametitle{Overview of sum factorization for Helmholtz equation}
\footnotesize{The 3D basis function is decomposed into the three 1D basis function:}
\begin{figure}
  \centering
  \includegraphics[width=4.7cm]{phi123.png}  
 \end{figure}
 \footnotesize{Using symmetry of D we will be able to expand:}
 \begin{figure}
  \centering
  \includegraphics[width=5cm]{D_entries.png}  
 \end{figure}
 \footnotesize{And the coefficient of the mass term is the following:}
\begin{figure}
  \centering
  \includegraphics[width=1.4cm]{defE.png}  
 \end{figure}
 }

\frame{\frametitle{Overview of sum factorization for Helmholtz equation}
\footnotesize{Expansion of the integrand:}
\begin{figure}
  \centering
  \includegraphics[width=4cm]{expansion.png}  
 \end{figure}
 \footnotesize{Factorization:}
 \begin{figure}
  \centering
  \includegraphics[width=3.5cm]{factorization.png}  
 \end{figure}
}

\frame{\frametitle{Overview of sum factorization for Helmholtz equation}
 \footnotesize{Optimized structure for integration of the factorized form:}
 \begin{figure}
  \centering
  \includegraphics[width=7cm]{Baux.png}  
 \end{figure}

 \footnotesize{where $B^{aux,1}_n\ (n=1,...,4)$ are arrays of integrals over $\xi_2$, which in turn are given as a function of $B^{aux,2}_m\ (m=1,...,8)$, arrays of integrals over $\xi_3$.}
}

\frame{\frametitle{Pseudocode on Helmholtz equation}
\footnotesize{
for each $\xi_1,\ i_3,j_3$ \\
\ \ \ \ set $B^{aux1}=0$ \\
\ \ \ \ for each $\xi_2$ \\
\ \ \ \ \ \ \ \ set $B^{aux2}=0$ \\
\ \ \ \ \ \ \ \ for each $\xi_3$ \\
\ \ \ \ \ \ \ \ \ \ \ \ accumulate for $B^{aux2}=B^{aux2}(\xi_2,i_3,j_3,\xi_1)$ \\
\ \ \ \ \ \ \ \ for each $i_2,\ j_2$ \\
\ \ \ \ \ \ \ \ \ \ \ \ accumulate for $B^{aux1}=B^{aux1}(i_2,j_2,i_3,j_3,\xi_1)$ \\
\ \ \ \ for each $i_2,\ j_2$ \\
\ \ \ \ \ \ \ \ for each $i_1,\ j_1$ \\
\ \ \ \ \ \ \ \ \ \ \ \ \ \ \ \ accumulate for $B_{IJ}=B^{aux}(i_1,i_2,i_3,j_1,j_2,j_3)$ \\
end}
}

\frame{\frametitle{Considerations for DPG}
\begin{itemize}
\item $p$:\ \ \ \ \ \ \ \ \ trial space approximation order.\\
$p+\Delta p$:\ enriched test space order
\item $k$ is the problem-dependent number of d.o.f.: $k=1$ for Poisson, $k=3$ for Primal Linear Elasticity, $k=6$ for Ultraweak Linear Elasticity
\item Element Gram matrix is of size $[k(p+\Delta p +1)]^2$ making it the largest array to compute within the \textbf{elem} subroutine.
\item Stiffness matrix is of size $k(p+\Delta p +1)\times k(p+1)$. Sum factorization is not applicable.
\item However, test and trial functions are already evaluated.
\item Their values may be reused to get Stiffness matrix and load vector within the Gram matrix's sum factorization algorithm
\end{itemize}
}
\section{Application example}
\frame{
  \frametitle{Elasticity Primal DPG - Sum Factorization on Gram Matrix}
  \footnotesize{
  \begin{itemize}
    \item $\Phi_I(\xi)\in V^{enr}(\Omega_e)=(H^1(\Omega^e))^3$\\ \\
    \pause
    \item $\Rightarrow\Phi_I(\xi)=(\phi_i(\xi))_\alpha=(\chi_{i1}(\xi_1)\chi_{i2}(\xi_2)\chi_{i3}(\xi_3))_\alpha$
    \pause
    \item $G^e_{IJ}=(\Phi_I,\Phi_J)_{V^{enr}(\Omega^e)}=(\Phi_I,\Phi_J)_{(H^1(\Omega^e))^3}$
    \pause
    \item $G^e_{IJ}=\int_{\Omega_e}\left(\nabla\Phi_I : \nabla\Phi_J\right+\Phi_I\cdot\Phi_J) d\Omega$
    \pause
    \item $G^e_{IJ}=\int_0^1\int_0^1\int_0^1\left[\left(\mathpzc{J}^{-T}\nabla_{\xi}\Phi_I\right) : \left(\mathpzc{J}^{-T}\nabla_{\xi}\Phi_J\right)+ \Phi_I\cdot\Phi_J\right] det(\mathpzc{J}) d\xi_3d\xi_2d\xi_1$ \\ \\
    \pause
    \item $G^e_{IJ}=\int_0^1\int_0^1\int_0^1\left[\left(\mathpzc{J}^{-T}\nabla_{\xi}\phi_i\right) \cdot \left(\mathpzc{J}^{-T}\nabla_{\xi}\phi_j\right)+ \phi_i\phi_j\right] \delta_{\alpha\beta}det(\mathpzc{J}) d\xi_3d\xi_2d\xi_1$ \\ \\
    \pause
    \item Same as Helmholtz case with $E=-det(\mathpzc{J})$ and taking into account the vector coordinates when multiplying by $\delta_{\alpha\beta}$
  \end{itemize}
  }
}
\frame{\frametitle{Pseudocode for primal elasticity including K and f}
\footnotesize{
for each $\xi_1,\ i_3,j_3$ \\
\ \ \ \ set $B^{aux1}=0$ \\
\ \ \ \ for each $\xi_2$ \\
\ \ \ \ \ \ \ \ set $B^{aux2}=0$ \\
\ \ \ \ \ \ \ \ for each $\xi_3$ \\
\ \ \ \ \ \ \ \ \ \ \ \ accumulate for $B^{aux2}=B^{aux2}(\xi_2,i_3,j_3,\xi_1)$ \\
\ \ \ \ \ \ \ \ \ \ \ \ \textcolor{red}{call getf,shap3H,geom3D} \\
\ \ \ \ \ \ \ \ \ \ \ \ \textcolor{red}{accumulate f} \\
\ \ \ \ \ \ \ \ \ \ \ \ \textcolor{red}{for each trial space function} \\
\ \ \ \ \ \ \ \ \ \ \ \ \ \ \ \ \textcolor{red}{accumulate K} \\
\ \ \ \ \ \ \ \ for each $i_2,\ j_2$ \\
\ \ \ \ \ \ \ \ \ \ \ \ accumulate for $B^{aux1}=B^{aux1}(i_2,j_2,i_3,j_3,\xi_1)$ \\
\ \ \ \ for each $i_2,\ j_2$ \\
\ \ \ \ \ \ \ \ for each $i_1,\ j_1$ \\
\ \ \ \ \ \ \ \ \ \ \ \ \textcolor{red}{for each $\alpha,\ \beta$=1,2,3; if $\alpha=\beta$} \\
\ \ \ \ \ \ \ \ \ \ \ \ \ \ \ \ accumulate for $G_{IJ}=B^{aux}(i_1,i_2,i_3,j_1,j_2,j_3)$ \\
end}
}


\section{Results} 

\frame{\frametitle{Linear Elasticity - Gram matrix computing time}
\begin{tabular}{|c|c|c|}
\hline
\textbf{p (trial)} & \textbf{Regular integration} & \textbf{Sum factorization} \\
\hline
1 & $2.724\times 10^{-3}$   & $2.794\times 10^{-3}$ \\
\hline
2 & $4.951\times 10^{-3}$   & $3.991\times 10^{-3}$ \\
\hline
3 & $2.445\times 10^{-2}$   & $7.048\times 10^{-3}$ \\
\hline
4 & $0.117166$   & $1.4758\times 10^{-2}$ \\
\hline
5 & $0.470836$   & $3.4624\times 10^{-2}$ \\
\hline
6 & $4.205$   & $3.8806\times 10^{-2}$ \\
\hline
\end{tabular}}


\frame{\frametitle{Linear Elasticity - Gram matrix computing time}
\begin{figure}
  \centering
  \includegraphics[width=8.5cm]{ela_gram1.eps}  
 \end{figure}
}


\frame{\frametitle{Linear Elasticity - elem() subroutine runtime}
\begin{tabular}{|c|c|c|}
\hline
\textbf{p (trial)} & \textbf{Regular integration} & \textbf{Sum factorization} \\
\hline
1 & $7.6\times 10^{-3}$   & $7.7\times 10^{-3}$ \\
\hline
2 & $2.42\times 10^{-2}$   & $1.98\times 10^{-2}$ \\
\hline
3 & $0.165$   & $0.1191$ \\
\hline
4 & $0.984$   & $0.797$ \\
\hline
5 & $6.90$   & $4.20$ \\
\hline
\end{tabular}}
\section{Conclusions}
\frame{\frametitle{Conclusions}
\begin{itemize}
\item An efficient technique for speeding up computation of DPG matrices for hexahedral elements is presented.
\item The integration time is decreased and in fact the whole runtime of the element subroutine shows considerable time saving for higher orders.
\item The integration time reduction also helps when computing errors and residuals, whether it is to evaluate error convergence or apply adaptive refinement.
\item The adaptation of this technique to more problems with different variational formulations will be investigated.
\end{itemize}
}

\frame{\frametitle{References}
\begin{enumerate}
\item Jason Patrick Kurtz, Fully Automatic hp-Adaptivity for Acoustic and Electromagnetic Scattering in Three Dimensions. The University of Texas at Austin. PhD dissertation, May 2007.
\item Leszek Demkowicz, Jay Gopalakrishnan; Discontinuous Petrov-Galerkin (DPG) Method, ICES REPORT 15-20, The Institute for Computational Engineering and Sciences, The University of Texas at Austin, October 2015.
\item Brendan Keith, Federico Fuentes, Leszek Demkowicz; The DPG methodology applied to different variational formulations of linear elasticity, Computer Methods in Applied Mechanics and Engineering, Volume 309, 1 September 2016, Pages 579-609.
\end{enumerate}

}

\frame{\ \ \ \ \ \ \ \LARGE{THANKS FOR YOUR ATTENTION}}

\end{document}

